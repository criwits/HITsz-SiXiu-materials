\nwtitle{《思想道德修养与法律基础》课程复习纲要}
\begin{center}
本文档由Hans Wan整理并开源至GitHub:\verb|https://github.com/criwits/HITsz-SiXiu-materials|
\end{center}

\section{人生的青春之问}

\subsection{人生观的主要内容及其之间的关系}
\begin{enumerate}
\item 人生观的主要内容包括人生目的、人生态度和人生价值。
\item 人生目的是指生活在一定历史条件下的人在人生实践中关于自身行为的根本指向和人生追求,是对“人为什么活着”这一人生根本问题的认识和回答,是人生观的核心。
\item 人生态度是指人们通过生活实践形成的对人生问题的一种稳定的心理倾向和精神状态,是人生观的重要内容。
\item 人生价值是指人的生命及其实践活动对于社会和个人所具有的作用和意义。
\item 总之,人生目的表明人的一生追求什么,人生态度表示以怎样的心态实现人生目标,人生价值判定一个具体人生的价值和意义。人生目的决定人们对待实际生活的基本态度和人生价值的评判标准,人生态度影响着人们对人生目的的持守和人生价值的评判,人生价值制约着人生目的和人生态度的选择。
\end{enumerate}

\subsection{人生价值的两个方面以及评价人生价值的科学方法}
\begin{enumerate}
\item 人生价值的两个方面是:
\begin{itemize}
\item 人生的自我价值,即个体的人生活动对自己的生存和发展所具有的价值,主要表现为对自身物质和精神需要的满足程度。
\item 人生的社会价值,即个体的实践活动对社会、他人所具有的价值。
\item 人生的自我价值和社会价值,既相互区别,又密切联系、相互依存,共同构成人生价值的矛盾统一体。
\end{itemize}
\item 评价人生价值的科学方法有:
\begin{itemize}
\item 坚持能力有大小与贡献须尽力相统一。考察一个人的人生价值,要把个人对社会的贡献同他的能力以及与能力相对应的职责联系起来。
\item 坚持物质贡献与精神贡献相统一。评价人生价值,既要看一个人对社会作出的物质贡献,也要看他对社会作出的精神贡献。
\item 坚持完善自身与贡献社会相统一。人生的社会价值是实现人生自我价值的基础。评价人生价值的大小应主要看一个人对社会所作的贡献,但这并不意味着要否定人生的自我价值。
\end{itemize}
\end{enumerate}

\subsection{辩证对待人生矛盾}
人生中矛盾是无处不在的。我们需要科学认识实际生活中的各种问题,勇敢面对和正确处理各种人生矛盾。
\begin{enumerate}
\item 树立正确的幸福观。我们应该思考,什么是人生的真正幸福,我们应该追求什么样的幸福,通过什么样的方式实现幸福。我们也需要清楚地认识到,实现幸福离不开一定的物质条件,更不能把自己的幸福建立在损害社会整体利益和他人利益的基础之上。只有在为社会做贡献、为他人服务的过程中,我们才能产生更大的幸福感。
\item 树立正确的得失观。我们要以积极进取的态度去面对生活中的成败得失,使一时的失败成为人生的财富而不是人生的包袱。我们不要拘泥于个人利益的得失,不要满足于一时的“得”,也不要惧怕一时的“失”。
\item 树立正确的苦乐观。苦与乐既对立又统一,在一定条件下可以相互转化。我们一块准确把我苦与乐的辩证关系,努力做迎难而上、艰苦奋斗的开拓者。
\item 树立正确的顺逆观。我们要善于利用顺境,勇于正视逆境和战胜逆境,才能够实现自己的人生价值。
\item 树立正确的生死观。生与死是贯穿人生始终的一对基本矛盾,我们应该珍惜韶华,在服务人民、投身民族复兴的伟大事业中开发出生命所蕴含的巨大潜能,努力给有限的个体生命赋予更有价值的意义。
\item 树立正确的荣辱观。我们应该坚持“八荣八耻”,明确在纷繁复杂的社会生活中应当坚持和提倡什么、反对和抵制什么,从而为自身判断行为得失,做出道德选择,确定价值取向提供基本的价值准则和行为规范。
\end{enumerate}

\section{坚定理想信念}

\subsection{理想信念是“精神之钙”}
理想指引方向,信念决定成败。理想信念是人生发展的内在动力。我们作为大学生,在提高知识水平。增强实践才干的同时,更要坚定崇高的理想信念。
\begin{enumerate}
\item 理想信念昭示奋斗目标。我们只有树立起崇高的理想信念,才能够解答好人生的意义、奋斗的价值以及做什么样的人等重要的人生课题。
\item 理想信念提供前进动力。大学时期确立起理想信念,对今后的人生之路将产生重要影响,甚至会影响终身。例如,大学生人生目标的确立、生活态度的形成、知识才能的丰富、发展方向的设定、工作岗位的选择,以及如何择友、如何面对挫折、如何克服困难等问题的解决,都离不开坚定的理想信念。
\item 理想信念提高精神境界。在追求理想和实现理想的过程当中,我们会不断面对各种挑战、抵御各种诱惑、突破各种局限、克服各种困难,而这个过程也是我们的精神境界从狭隘走向高远、从空虚走向充实、从犹疑走向执着的过程。
\end{enumerate}

大学生只有树立崇高的理想信念,才能激发起为民族复兴和人民幸福而发奋学习的强烈责任感和使命感。

\subsection{为什么要信仰马克思主义}
马克思主义作为我们立党立国的根本指导思想,是近代以来中国发展的必然结果,是中国人民长期探索的历史选择,也是由马克思主义严密的科学体系、鲜明的阶级立场和巨大的实践指导作用决定的。
\begin{enumerate}
\item 马克思主义体现了科学性和革命性的统一。马克思主义深刻地揭示了自然界、人类社会和人类思维发展的普遍规律,揭示了事物的本质、内在联系和发展规律,是“伟大的认识工具”,是人们观察世界、分析问题的有力思想武器。
\item 马克思主义具有鲜明的实践品格。马克思主义不仅致力于科学地解释世界,而且致力于积极地改变世界。在人类思想史上,还没有一种理论像马克思主义那样对人类文明进步产生如此广泛而巨大的影响。
\item 马克思主义具有持久生命力。作为一个开放的理论体系,马克思主义不断地吸收、提炼人类创造的一切科学知识和文明成果,并将其运用于推进历史的进步。无论时代如何变迁、科学如何进步,马克思主义依然占据着真理和道义的制高点,仍然具有强大持久的生命活力。
\end{enumerate}

大学生只有树立马克思主义的科学信仰,才能真正确立崇高的理想信念,在错综复杂的社会现象中看清本质、明确方向,为服务人民、奉献社会作出更大的贡献。

\subsection{中国特色社会主义是我们的“共同理想”,它与我们的“远大理想”之间有怎样的关系}
\begin{enumerate}
\item 实现中国特色社会主义是我们的共同理想,实现共产主义是我们的远大理想。
\item 我们要牢固确立在中国共产党领导下走中国特色社会主义道路、为实现中华民族伟大复兴而奋斗的共同理想和坚定信念。
\item 而共产主义远大理想的实现是一个漫长、艰辛的历史过程,也需要我们一代又一代人的不懈奋斗和接续努力。
\item 实现共产主义是我们的远大理想,坚持和发展中国特色社会主义,就是向着远大理想所进行的实实在在的努力。
\end{enumerate}

走好新时代的长征路,大学生要不断增强中国特色社会主义道路自信、理论自信、制度自信、文化自信,自觉做共产主义远大理想和中国特色社会主义共同理想的坚定信仰者、忠实实践者,为崇高理想信念而矢志奋斗。

\subsection{个人理想与社会理想的统一}
个人理想指处于一定历史条件和社会关系中的个体对于自己未来的物质生活、精神生活所产生的种种向往和追求。社会思想指社会集体乃至社会全体成员的共同思想,即在全社会占主导地位的共同奋斗目标。社会理想与个人理想不是彼此孤立的,而是相互联系、相互影响和相互制约的。
\begin{enumerate}
\item 个人理想以社会理想为指引。个人理想的确立要以社会理想为主导,个人理想的实现依赖于社会理想的实现。我们的个人理想只有同国家的前途、民族的命运相结合,个人的向往和追求只有同社会的需要和人民的利益相一致,才可能变为现实。
\item 社会理想是对个人理想的凝练和升华。社会理想不是凭空产生的,也不是由外在力量强加的,而是建立在众人的个人理想基础之上。
\end{enumerate}

我们作为大学生,要在社会思想的指引下,珍惜韶华、奋发有为,勇于追求个人理想,在实现社会理想的过程中努力实现个人理想。

\section{弘扬中国精神}
\subsection{民族精神的内涵(“四个伟大”)}
民族精神是一个民族在长期共同生活和社会实践中形成的,为本民族大多数成员所认同的价值取向、思维方式、道德规范、精神气质的总和,是一个民族赖以生存和发展的精神支柱。
\begin{enumerate}
\item 伟大创造精神。中国人民自古以来始终辛勤劳作、发明创造。我国产生了老子、孔子、庄子、孟子等伟大思想巨匠,发明了造纸术、火药、印刷术等伟大科技成果,创作了《诗经》、楚辞、汉赋等伟大文艺作品,建设了万里长城、布达拉宫、故宫等伟大工程。而今天,中国人民的伟大创造精神正在前所未有地迸发出来,推动我国经济日新月异地向前发展。
\item 伟大奋斗精神。在几千年历史长河中,中国人民始终革故鼎新、自强不息。中国人民自古就明白,要幸福就要奋斗;而今天,只要13亿多中国人民始终发扬这种伟大奋斗精神,就一定能够达到创造人民更加美好生活的宏伟目标。
\item 伟大团结精神。自古以来,中国人民始终团结一心、同舟共济。今天,中国取得的令世人瞩目的发展成就,更是全国各族人民同心同德、同向努力的结果。
\item 伟大梦想精神。中国人民自古以来始终心怀梦想、不懈追求。中国人民相信,山再高,往上攀,总能登顶;路再长,走下去,定能到达。这种伟大的梦想精神,是支持我们不断奋斗,奋勇向前的精神力量。
\end{enumerate}

勤劳勇敢的中国人民培育、继承、发展起来的以爱国主义为核心的伟大民族精神,是坚定中国特色社会主义道路自信、理论自信、制度自信、文化自信地底气,是中华民族风雨无阻、高歌行进的根本力量。

\subsection{爱国主义的基本内涵及其时代要求}
\begin{enumerate}
\item 爱国主义体现了人们对自己祖国的深厚感情,揭示了个人对祖国的依存关系,是人们对自己家园以及民族和文化的归属感、认同感、尊严感和荣誉感的统一。它的主要内涵有:
\begin{itemize}
\item 爱祖国的大好河山。祖国的河山在人们的心中占据着至高无上的地位,它不只是自然风光,更是主权、财富、民族发展和进步的基本载体。每一个爱国者都应该把“保我国土”“爱我家乡”、维护祖国领土的完整和统一作为自己的神圣使命和义不容辞的责任。
\item 爱自己的骨肉同胞。中华民族的利益是我国各族人民的共同利益、长远利益和最高利益,这种利益高于各个民族内部的、局部的、暂时的利益。我们要始终坚持以人民为中心的立场,始终紧紧地同人民群众站在一起。
\item 爱祖国的灿烂文化。文化是一个国家、一个民族的灵魂。我们要在充分理解和尊重的基础之上,积极推动祖国优良历史文化传统的传承和发展。
\item 爱自己的国家。我们要拥护国家的基本制度,遵守国家的宪法法律,维护国家安全和统一,捍卫国家的利益,为国家繁荣发展贡献自己的力量,这些都是爱国主义的基本要求。
\end{itemize}
\item 新时代的爱国主义,既承接了中华民族的爱国主义优良传统,又体现了鲜明的时代特征,内涵更加丰富。新时代的爱国主义要求:
\begin{itemize}
\item 坚持爱国主义和社会主义相统一。在当代中国,爱国主义首先体现在对社会主义的热爱上。社会主义制度的建立,为中国的繁荣发展提供了可靠的保障。爱国主义与爱社会主义的统一是中国历史发展的必然结果。而坚定拥护中国共产党的领导,是中华民族走向复兴、中国特色社会主义事业走向成功的必然要求,也是新时代爱国主义的必然要求。
\item 维护祖国统一和民族团结。维护和推进祖国统一,是中华民族走向伟大复兴的题中之义;而中华民族和各民族的关系,是一个大家庭和家庭成员的关系。只有维护好祖国统一和民族团结,我们才能弘扬新时代的爱国精神。
\item 尊重和传承中华民族历史和文化。对祖国悠久历史、深厚文化的理解和接受,是人们爱国主义情感培育和发展的重要条件。我们必须尊重和传承中华民族历史和文化,以时代精神激活中华民族优秀传统文化的生命力,增强文化自信,才能增强自己作为中国人的骨气和底气。
\item 必须坚持立足民族又面向世界。中国的命运与世界的命运紧密相关,弘扬新时代的爱国主义,必须既坚持立足民族,维护国家发展的主体性,又必须面向世界,构建人类命运共同体。
\end{itemize}
\end{enumerate}

\subsection{改革创新是时代要求}
改革创新是社会发展的重要动力,坚持改革创新是新时代的迫切要求。
\begin{enumerate}
\item 创新始终是推动人类社会发展的第一动力。从某种意义上说,创新决定着世界政治经济力量对比的变化,也决定着各国各民族的前途命运。
\item 创新能力是当今国际竞争新优势的集中体现。今天,国际竞争的新优势越来越集中体现在创新能力上,创新战略竞争在综合国力竞争中的地位日益重要。
\item 改革创新是我国赢得未来的必然要求。在新一轮科技革命和产业变革中,我国能否在未来发展中后来居上、弯道超车,主要就看能否在创新驱动发展上迈出实实在在的步伐。
\end{enumerate}

“聪者听于无声,明者见于未形。”大学生要自觉树立敢为天下先的志向和信心,敢于担当、勇于超越,在攻坚克难中追求卓越,在改革创新中引领世界潮流。

\subsection{做改革创新“生力军”}
新时代的大学生应当以时代使命为己任,把握时代的脉搏,迎接时代的挑战,增强创新创造的能力和本领,勇做改革创新的实践者,将弘扬改革创新精神贯穿于实践中、体现在行动上。
\begin{enumerate}
\item 树立改革创新的自觉意识。
\begin{itemize}
\item 增强改革创新的责任感。我们要不断增强以改革创新推动社会进步,在改革创新中奉献服务社会、实现人生价值的崇高责任感和使命感,以时不我待、只争朝夕的紧迫感投身改革创新的实践中。
\item 树立大胆探索未知领域的信心。我们应是常为新、敢创造的,理当锐意创新创造,不等待、不观望、不懈怠,勇做改革创新的生力军。
\end{itemize}
\item 增强改革创新的能力本领。
\begin{itemize}
\item 夯实创新基础。我们应从扎实系统的专业知识学习起步入手,而不能好高骛远,空谈改革,坐论创新。
\item 培养创新思维。我们在专业学习与社会实践中应自觉培养创新型思维,勤于思考,善于发现,勇于创新。
\item 投身创新实践。我们应当在全面深化改革的伟大实践中深深体悟改革创新精神,增强改革创新的意识,锤炼改革创新的意志,增强改革创新的能力本领,勇做改革创新的实践者和生力军。
\end{itemize}
\end{enumerate}

大学生应当珍惜人生中最具有创新创造力的宝贵时期,有敢为人先、开拓进取的锐气,有逢山开路、遇河架桥的意志,在创新创造中不断积累经验、取得成果、演绎精彩。

\section{践行社会主义核心价值观}
\subsection{社会主义核心价值观的基本内容}
\begin{enumerate}
\item 富强、民主、文明、和谐。这一价值追求回答了我们要建设什么样的国家的问题,揭示了当代中国在经济发展、政治文明、文化繁荣、社会进步等方面的价值目标,从国家层面标注了社会主义核心价值观的时代刻度。
\item 自由、平等、公正、法治。这一价值追求回答了我们要建设什么样的社会的重大问题,与实现国家治理体系和治理能力现代化的要求相契合,揭示了社会主义社会发展的价值取向。
\item 爱国、敬业、诚信、友善。这一价值追求回答了我们要培育什么样的公民的重大问题,涵盖了社会公德、职业道德、家庭美德、个人品德等方面,是每一个公民都应当遵守的道德规范。
\end{enumerate}

\subsection{坚定核心价值观自信的原因}
坚定的核心价值观自信,是中国特色社会主义道路自信、理论自信、制度自信和文化自信的价值内核。
\begin{enumerate}
\item 社会主义核心价值观有深厚的历史底蕴。社会主义核心价值观不是无源之水、无本之木,它深深地根植于中华传统优秀文化,更是对中华优秀传统文化的继承和升华。
\item 社会主义核心价值观有实际的现实基础。中国特色社会主义是社会主义核心价值观的实践依据,也以无可辩驳的事实生动展示着社会主义核心价值观的生机活力。而社会主义核心价值观则是当代中国精神的集中呈现,是中国特色社会主义本质规定的价值表达。
\item 社会主义核心价值观有强大的道义力量。社会主义核心价值观以其先进性、人民性和真实性而居于人类社会的价值至高点,具有强大的道义力量和感召力,是我国社会主义的内在精神之魂。
\end{enumerate}

社会主义核心价值观具有无比的优越性和巨大意义,我们应该坚定核心价值观自信,自觉以社会主义核心价值观来引领我们接力前行。

\subsection{大学生如何做社会主义核心价值观的积极践行者}
大学生需要始终走在时代前列,成为社会主义核心价值观的坚定信仰者、积极传播者、模范践行者。
\begin{enumerate}
\item 扣好人生的扣子。青年的价值取向决定了未来整个社会的价值取向,而青年又处在价值观形成和确立的时期,抓好这一时期的价值观养成十分重要。此外,大学生的健康成长成才更加需要正确价值观的引领。
\item 勤学修德明辨笃实。对于大学生而言,切实做到这四点,才能使社会主义核心价值观成为一言一行的基本遵循。
\begin{itemize}
\item 勤学。大学生要注重把所学知识内化于心,形成自己的见解,专攻博览,努力掌握为祖国、为人民服务的真才实学。
\item 修德。一个人只有明大德、守公德、严私德,其才方能用得其所。
\item 明辨。培育和践行社会主义核心价值观,要增强自己的价值判断力和道德责任感,辨别什么是真善美、什么是假恶丑,自觉做到常秀善德、常怀善念、常做善举。
\item 笃实。于实处用力,从知行合一上下功夫,核心价值观才能内化为人们的精神追求,外化为人们的自觉行动。青年要把艰苦环境作为磨练自己的机遇,把小事当作大事看,一步一个脚印往前走。
\end{itemize}
\end{enumerate}

积极践行社会主义核心价值观,既要目标高远,又要脚踏实地,勤学以增智、修德以立身、明辨以正心、笃实以为功。

\section{明大德、守公德、严私德}
\subsection{马克思主义道德的起源与本质}
\begin{enumerate}
\item 马克思主义道德观认为道德的起源是多方面的:
\begin{itemize}
\item 劳动是道德起源的首要前提。劳动将人和动物区分开来,创造了人、社会和社会关系,也创造了道德。其中,劳动创造了道德主体,劳动分工合作则使得利益关系日渐清晰,自由、责任等内容的道德逐步得到确认。因此,劳动创造了人类社会,是道德起源的第一个历史前提。
\item 社会关系是道德赖以产生的客观条件。可以说,正是社会关系的形成和发展产生了调节各种关系特别是利益关系的需要,道德恰恰是适应社会关系调节的需要而产生的。
\item 人的自我意识是道德产生的主观条件。意识是道德产生的思想认识前提,只有当人意识到自我在社会中的角色和地位,意识到自我与他人或集体不同的利益关系,并由此产生调节利益的迫切要求时,道德才得以产生。
\end{itemize}
\item 马克思主义道德观认为道德的本质是多层次的:
\begin{itemize}
\item 道德是反映社会经济关系的特殊意识形态。道德的产生、发展和变化,归根结底根源于社会经济关系。
\item 道德是社会利益关系的特殊调节方式。道德是一种调整人与人、人与社会、人与自然以及人与自身之间关系的特殊的行为规范。
\item 道德是一种实践精神。道德是一种以指导人的行为为目的、以形成人的正确行为方式为内容的的精神,在本质上是知行合一的。
\end{itemize}
\end{enumerate}

\subsection{中国革命道德的主要内容和当代价值}
中国革命道德,是对中华传统美德的延续和发展。传承和发扬中国革命道德,是弘扬中华传统美德的应有之义,是加强社会主义道德建设的客观需要,也是激励大学生锤炼优良道德品质的必然要求。
\begin{enumerate}
\item 中国革命道德具有丰富而独特的内涵,其可以概括如下:
\begin{itemize}
\item 为实现社会主义和共产主义理想而奋斗。坚持社会主义、共产主义理想和信念的不屈不挠的精神,是革命道德的灵魂。我们的革命先烈之所以能够排除万难、坚持斗争、无私无畏、不怕牺牲,就是因为他们有坚定的社会主义、共产主义的理想和信念。
\item 全心全意为人民服务。可以说,全心全意为人民服务作为贯穿中国革命道德始终的一根红线,是中国共产党在中国革命实践中的一个伟大创造,对中国的革命、建设、改革事业,产生了极其重大的推动作用。
\item 始终把革命利益放在首位。始终把革命利益放在首位,可以极大地激发革命者为集体而献身的斗志,使革命队伍形成前所未有的向心力和凝聚力,也使革命事业不断蓬勃地向前发展。
\item 树立社会新风,建立新型人际关系。引导建立新型家庭关系和培养良好家风,对于提升人民群众的文明水准和道德规范,树立社会新风尚,发挥着重要的作用。
\item 修身自律,保持节操。具体而言,中国革命道德要求以中国革命事业为重,严于律己,谦虚谨慎;淡泊名利,清正廉洁;襟怀坦白,光明磊落;始终保持高风亮节,展现出高尚的人格力量。
\end{itemize}
\item 中国革命道德内容丰富,历久弥新。它在今天对于我们走好新时代的长征路,实现中华民族伟大复兴仍然具有极其重要的现实意义。中国革命道德:
\begin{itemize}
\item 有利于加强和巩固社会主义和共产主义的理想信念。弘扬中国革命道德,有利于树立和培养人民群众的理想信念,有利于坚持和发展中国特色社会主义道路。
\item 有利于培育和践行社会主义核心价值观。中国革命道德,是先进价值观在道德领域的集中体现。继承和弘扬中国革命道德,对于帮助人们深刻理解社会主义核心价值观的科学内涵和历史底蕴,增强价值观认同,为中国特色社会主义事业提供攻坚克难的强大精神支撑,具有重要意义。
\item 有利于引导人们树立正确的价值观。在今天,发扬光大革命道德,能够引导人们正确对待个人利益和社会利益、国家利益,能够帮助人们在深刻把握历史、认识社会、审视人生的基础上,积极投入到决胜全面建成小康社会、夺取新时代中国特色社会主义伟大胜利的新征程。
\item 有利于培育良好的社会道德风尚。解决道德领域出现的突出问题,要充分发挥革命道德的精神力量,培育良好的社会道德风尚,树立浩然正气,凝聚崇德向善的正能量。
\end{itemize}
\end{enumerate}

作为大学生,我们需要深入了解中国社会和中国革命的历史,了解中国共产党人带领广大人民群众进行革命斗争的艰苦实践,深刻体会中国革命道德的内涵,并传承和发扬中国革命道德的当代价值。

\subsection{社会主义道德建设的核心和原则}
社会主义道德建设是社会主义文化建设的重要内容。了解社会主义道德的核心和原则,对于大学生践行社会主义道德、锤炼道德品质具有重要意义。
\begin{enumerate}
\item 为人民服务是社会主义道德的核心。为什么人服务是道德的核心问题,决定并体现着道德建设的根本性质和发展方向,规定并制约着道德领域中的所有道德现象。
\begin{itemize}
\item 为人民服务是社会主义经济基础和人际关系的客观要求。在我国,以公有制为主体和以按劳分配为主体,是为人民服务的根本制度保证,在此基础上逐步形成的团结互助、平等友爱、共同进步的人际关系,是为人民服务的基础。
\item 为人民服务是社会主义市场经济健康发展的要求。社会主义市场经济,不仅不排斥为社会和他人服务,而且需要通过服务甚至是优质服务,才能实现市场主体的利益。
\item 为人民服务是先进性要求和广泛性要求的统一。一个有道德的人、一个具有为人民服务意识的人,必定会有为他人服务、为社会献身的精神,会时时处处想到别人,想到社会,想到国家,从而能够推己及人、与人为善,服务他人、奉献社会。
\end{itemize}
\item 集体主义是社会主义道德的原则。长期以来,集体主义已经成为调节国家利益、社会整体利益和个人利益关系的基本原则。
\begin{itemize}
\item 集体主义强调国家利益、社会整体利益和个人利益的辩证统一。集体与个人,即‘统’与‘分’,是相互作用、相互依赖、互为前提的辩证统一关系。只有使二者有机地结合起来,才能使生产力保持旺盛的发展势头,偏废任何一方,都会造成大损失。
\item 集体主义强调国家利益、社会整体利益高于个人利益。社会主义集体主义之所以强调个人利益要服从国家利益、社会整体利益,归根到底,既是为了维护国家、社会的共同利益,最终也是为了维护个人的根本利益和长远利益。
\item 集体主义重视和保障个人的正当利益。集体主义促进和保障个人正当利益的实现,使个人的才能、价值得到充分的发挥。这不但与集体主义不矛盾,而且正是集体主义思想的应有之义。
\end{itemize}
\end{enumerate}

\subsection{网络生活中的道德要求}
从本质上说,网络交往仍然是人与人的现实交往,网络生活也是人的真实生活。网络生活中的道德要求,是人们在网络生活中为了维护正常的网络公共秩序需要共同遵守的基本道德准则,是社会公德在网络空间的运用和扩展。
\begin{enumerate}
\item 正确使用网络工具。大学生应当正确使用网络,提高信息的获取能力,加强信息的辨识能力,增进信息的应用能力,使网络成为开阔视野、提高能力的重要工具。
\item 健康进行网络交往。大学生应通过网络开展健康有益的人际交往,树立自我保护意识,不要轻易相信网友,避免受骗上当,避免给自己的人身和财产安全带来危害。
\item 自觉避免沉迷网络。大学生应当合理安排上网时间,约束上网行为,避免沉迷网络。
\item 加强网络道德自律。网络空间同现实社会一样,既要提倡自由,也要保持秩序。大学生应当在网络生活中培养自律精神,在缺少外在监督的网络空间里,做到自律而“不逾矩”,促进网络生活的健康与和谐。
\item 积极引导网络舆论。作为新时代的大学生,应当带头引导网络舆论,对模糊认识要及时廓清,对怨气怨言要及时化解,对错误看法要及时引导和纠正,积极营造清朗网络空间。
\end{enumerate}

\subsection{大学生如何树立正确的择业观和创业观}
树立正确的择业观和创业观,对于大学生顺利走进职业生活具有重要的现实意义。
\begin{enumerate}
\item 树立崇高的职业理想。我们应该学习和追求青年马克思的崇高的职业理想,选择最能为人类而工作的职业,为社会的发展做出贡献。
\item 服从社会发展的需要。大学生应该积极响应国家号召,适应社会发展需求,面向基层、面向国家建设第一线去选择自己未来的职业,为经济社会发展贡献智慧和力量。
\item 做好充分的择业准备。大学生应认识到,任何一名劳动者,无论从事的劳动技术含量如何,只要兢兢业业、精益求精,就一定能够造就闪光的人生。
\item 培养创业的勇气和能力。大学生不仅要树立正确的择业观,还应当树立正确的创业观。要有积极创业的思想准备,培养敢于创业的勇气,充分考虑自身的条件,积极关注经济社会发展的趋势,为今后自主创业打下良好的基础。
\end{enumerate}

\subsection{怎样积极投身从德向善的道德实践}
大学生投身崇德向善的道德实践,就要向道德模范学习,培养志愿服务精神,大力弘扬时代新风,强化社会责任意识、规则意识、奉献意识。
\begin{enumerate}
\item 向道德模范学习。大学生要向道德模范学习,崇德向善、见贤思齐,弘扬真善美,传播正能量,积极从道德模范身上获取前进的动力,做社会良知的守望者、积极传播者和践行者。
\item 参与志愿服务活动。志愿服务已经成为大学生参与社会实践、成长成才的重要舞台,成为大学生关爱他人、传播青春正能量的重要途径。大学生应带头学雷锋,做雷锋精神的种子,把雷锋精神广播在祖国大地上。
\item 引领社会风尚。大学生投身崇德向善的道德实践,要弘扬真善美、贬斥假恶丑,做社会主义道德的示范者和引领者,促成知荣辱、讲正气、作奉献、促和谐的社会风尚。
\begin{itemize}
\item 讲正气。讲正气,就是坚持真理、坚持原则,坚持同一切歪风邪气作斗争。大学生须有一腔浩然正气,才能无所畏惧地前进,才能不屈不挠地为国家、为社会建功立业。
\item 作奉献。选择奉献也就选择了高尚。“德厚者流光”,大学生要在奉献社会中积极发光发热,使我们的社会更加美好和幸福。
\item 促和谐。大学生要用和谐的态度对待人生实践,使崇尚和谐、维护和谐内化为自己的思想意识和行为习惯,推动人与人之间、人与社会之间融洽相处,实现人与自然之间友好共生。
\end{itemize}
\end{enumerate}

社会文明状况是社会风尚的重要体现。大学生要以高度的主人翁精神,积极参与各种精神文明创建活动,为家庭谋幸福、为他人送温暖、为社会作贡献,不断引领社会风尚,提升道德品质。

\section{尊法、学法、守法、用法}
\subsection{法律的含义与特征}
\subsection{我国社会主义法律的本质特征}
\subsection{宪法的地位、基本原则和制度}
\subsection{全面依法治国的十六字方针}
\subsection{走中国特色社会主义法治道路的“五个坚持”}
\subsection{法治思维的内容、基本内涵及其培养}
