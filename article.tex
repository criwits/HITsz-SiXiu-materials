\nwtitle{《思想道德修养与法律基础》课程复习纲要}

\section{人生的青春之问}
\subsection{人生观的主要内容及其之间的关系}
\begin{enumerate}
\item 人生观的主要内容包括人生目的、人生态度和人生价值。
\item 人生目的是指生活在一定历史条件下的人在人生实践中关于自身行为的根本指向和人生追求,是对“人为什么活着”这一人生根本问题的认识和回答,是人生观的核心。
\item 人生态度是指人们通过生活实践形成的对人生问题的一种稳定的心理倾向和精神状态,是人生观的重要内容。
\item 人生价值是指人的生命及其实践活动对于社会和个人所具有的作用和意义。
\item 总之,人生目的表明人的一生追求什么,人生态度表示以怎样的心态实现人生目标,人生价值判定一个具体人生的价值和意义。人生目的决定人们对待实际生活的基本态度和人生价值的评判标准,人生态度影响着人们对人生目的的持守和人生价值的评判,人生价值制约着人生目的和人生态度的选择。
\end{enumerate}
\subsection{人生价值的两个方面以及评价人生价值的科学方法}
\begin{enumerate}
\item 人生价值的两个方面是:
\begin{itemize}
\item 人生的自我价值,即个体的人生活动对自己的生存和发展所具有的价值,主要表现为对自身物质和精神需要的满足程度。
\item 人生的社会价值,即个体的实践活动对社会、他人所具有的价值。
\item 人生的自我价值和社会价值,既相互区别,又密切联系、相互依存,共同构成人生价值的矛盾统一体。
\end{itemize}

\end{enumerate}
\subsection{辩证对待人生矛盾}

\section{坚定理想信念}
\subsection{理想信念是“精神之钙”}
\subsection{为什么要信仰马克思主义}
\subsection{中国特色社会主义是我们的“共同理想”,它与我们的“远大理想”之间有怎样的关系}
\subsection{个人理想与社会理想的统一}

\section{弘扬中国精神}
\subsection{民族精神的内涵(“四个伟大”)}
\subsection{爱国主义的基本内涵及其时代要求}
\subsection{改革创新是时代要求}
\subsection{做改革创新“生力军”}

\section{践行社会主义核心价值观}
\subsection{社会主义核心价值观的基本内容}
\subsection{坚定核心价值观自信的原因}
\subsection{大学生如何做社会主义核心价值观的积极践行者}

\section{明大德、守公德、严私德}
\subsection{马克思主义道德的起源与本质}
\subsection{中国革命道德的主要内容和当代价值}
\subsection{社会主义道德建设的核心和原则}
\subsection{网络生活中的道德要求}
\subsection{大学生如何树立正确的择业观和创业观}
\subsection{怎样积极投身从德向善的道德实践}

\section{尊法、学法、守法、用法}
\subsection{法律的含义与特征}
\subsection{我国社会主义法律的本质特征}
\subsection{宪法的地位、基本原则和制度}
\subsection{全面依法治国的十六字方针}
\subsection{走中国特色社会主义法治道路的“五个坚持”}
\subsection{法治思维的内容、基本内涵及其培养}
